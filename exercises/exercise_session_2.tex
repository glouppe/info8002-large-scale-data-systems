% Include the header of the document.
\input{template_header}

\title{Large-Scale Distributed Systems: Exercise Session 2}
\maketitle

\section{Introduction}
\label{sec:introduction}

Broadcast communication is used to disseminate information among a set of processes and differ according to the reliability of the dissemination.  In a traditional \emph{server-client} architecture, interactions are often established between two processes. In this architecture, a server exposes some kind of interface to the client, which in turn exploits this interface to exchange some state with all other participating clients (e.g., as in a multiplayer game). These interactions are supported by a \emph{direct link} between the server and all its clients. Meaning, there is a dedicated \emph{point-to-point} link for every client with the centralized server.\\

Of course, this approach has several advantages and disadvantages. The main, albeit practical advantage of this paradigm is that only a single machine (or a set of machines that appear as a single service) is responsible for maintaining some state. This enables system engineers to fine-tune their service without relying on other clients which may have non-optimal connections, or have even malicious intents. This is actually what we observe nowadays. We are moving from a once idealized notion of \emph{peer-to-peer} content delivery systems to \emph{cloud} based architectures (NetFlix, YouTube, \ldots). Of course, a very high cost is associated with building such a system. However, it provides consistent performance because it does not have to rely on the \emph{quality} of participating nodes in the network. Nevertheless, it brings the traditional issues of introducing a single point of failure for the system. So a paradigm shift of \emph{expecting failure} in P2P computing towards \emph{preventing failure} can be reconsidered. Of course, these monolitic services still consist of several thousands of individual, but cooperating machine. So in a sense you still have to \emph{expect failure} within this architecture.\\

On the other hand, in \emph{multiparticipant systems} broadcasting is slightly more complex since all participating nodes have to agree on the message they collectively received. Additionaly, there is the issue how a message is efficiently distributed to all nodes in the network. Furthermore, one might as well be interested in the ordering of broadcasted messages. To accomplish this, it is useful to rely on \emph{broadcasting abstractions}.

\section{Exercises}
\label{sec:exercises}

\subsection*{Exercise 1}
\label{sec:exercise_1}

\emph{Exercise one goes here.}

\subsection*{Exercise 2}
\label{sec:exercise_2}

\emph{Exercise two goes here.}

\subsection*{Exercise 3}
\label{sec:exercise_3}

\emph{Exercise three goes here.}

\subsection*{Exercise 4}
\label{sec:exercise_4}

\emph{Exercise four goes here.}

\subsection*{Exercise 5}
\label{sec:exercise_5}

\emph{Exercise five goes here.}

\subsection*{Exercise 6}
\label{sec:exercise_6}

\emph{How does Ethernet (IEEE 802.3) relate to broadcasting? If Ethernet supports broadcasting, how does it ensure ensure reliable broadcasting? In the other case, how would you implement broadcasting using Ethernet? Note: One could apply the same mechanisms to WiFi due the the presence of a shared medium. However, Ethernet is a little bit more tricky since not all machines are connected to the same physical cable.}

% Include the footer of the document.
\input{template_footer}
